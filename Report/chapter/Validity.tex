\documentclass[../main.tex]{subfiles}


\begin{document}

\section{Validity}

Our approach has a high efficiency in finding clones, but there are still a number of threats to the validity:

\textbf{Statistical limitation.} STCD is a statiatical based tool, so each method needs to be big enough to get statistical data. In the actual testing, for methods less than 8 lines, the comparision is not reasonable, non-similar pairs can be taken as clones. So during the implementation of STCD, we ignored methods with less than 8 lines to reduce the false positives.

\textbf{Data Selection.} Both training data and test data are chosen from SWT source files, the good part of doing so is that machine learning method may learn the developer's behavior, however on the other hand, the performance of our tool is unknown for other files. Further research could be done by training and testing more files or projects. 

\textbf{Data Size.} The size of both training data and test data sets are small, we selected only 10 files in training and 10 files in testing, with actual clones no more than 20 in each file. 

\textbf{Manual Selection.} Manual selection leads to false negatives. As discussed in the test results, by ruling out most unrelated comparisons with a low threshold, we got 2 actual clones ruled out, i.e. false negatives. It is unsure if there are more false negatives like this, so our result may differ from the truth.

\textbf{Ambiguous Match.} Ambiguous match is realized by threshold. First is the threshold of 0.7 used in the pre-processing of variable names. If variable names have a similarity higher than 0.7, they are considered the same and compared. However this may not always be true. By a threshold of 0.7 we may either put unrelated variables together, or take similar variables as different.

Second is the final threshold of 0.87. A lower threshold will result in a lower precision while a higher threshold will result in a lower recall. But that is not absolute. It may either rule out actual clones or print out unrelated pairs.

\textbf{Extra Time Cost.} In the first step of implementation, method declaration, we used ASTParser Tool to split the file into methods and get the information of each method. ASTParser Tool is based on Abstract Syntax Trees, by transforming the original file into trees, extra time cost is introduced, which is not needed. A simpler method(e.g. regular expression) can be used to reduce time in this part.


\end{document}